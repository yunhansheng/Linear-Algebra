vector space → subspace

(无其他特定结构的)

basis (如何刻画vector space) → linearly dependency/span

定理:等价刻画

transformation → isomorphism

定理:等价定义(invertible)
定理:L(v,w)也构成vecotr space
定理:vector space被基底唯一刻画
    推论:n维vector space均同构
    
(矩阵表示+换底)

--------------------------------------------------

(有特定内积结构的vector space:Euclidean Space)
内积
    1. 长度
    2. schwarz-inequality(不同euclidean space的视角)角度 orthogonal
    3. 距离

正交、正交规范基
    1. 正交化 定理:正交基存在性
    2. 寻找正交投影
    
isomorphism (vector space isomorphism + inner product) (consequence, 3d elementary geometry)

(矩阵表示+换底)

--------------------------------------------------

linear function (form)
bilinear function (form), symmetric (eg. inner product)
quadratic form, positive (inner product) 标准型化简

(矩阵表示+换底)
