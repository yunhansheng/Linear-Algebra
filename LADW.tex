\documentclass[hidelinks]{article}

\usepackage{xeCJK}
\usepackage[a4paper,top=3cm,bottom=3cm,left=3cm,right=3cm,marginparwidth=1.75cm]{geometry}
\usepackage{amsmath,amsthm,amsfonts}
\usepackage[utf8]{inputenc}
\usepackage{amssymb}
\usepackage[mathscr]{eucal}
\usepackage{graphicx}

\setcounter{section}{+0}
\setcounter{subsection}{+0}

\theoremstyle{definition}
\newtheorem*{defin}{Def}
\theoremstyle{plain}
\newtheorem{theorem}{Thm}[section]
\newtheorem{proposition}[theorem]{Prop}
\theoremstyle{remark}
\newtheorem*{remark}{Remark}

\usepackage{hyperref}
\hypersetup{colorlinks=false}

\DeclareMathOperator{\range}{ran}



\begin{document}
\begin{center}
\LARGE\textbf{A Streamlined Version of LADW}\newline
\end{center}

\section{Basic Notations}

\subsection{Vector Spaces}
\begin{defin}vector space\newline
set with 2 operations satisfying: \newline
\indent A1. add. commutativity
\indent A2. add. associativity
\indent A3. add. identity
\indent A4. add. inverse \newline
\indent M1. mult. inverse
\indent M2. mult. associativity \newline
\indent D. mutual distributive
\end{defin}

\subsection{Linear combinations, bases}
\begin{defin}~\\
basis \newline
generating/complete system \newline
linearly (in)dependent system
\end{defin}

\begin{proposition} (alternative def. of linearly (in)dependent)~\\

\end{proposition}

\begin{proposition} (alternative def. of basis)~\\
$|\textrm{generating system}|>|basis|$
\end{proposition}

\subsection{Linear Transformations. Matrix-vector multiplication}

\begin{defin}
linear transformation
\end{defin}

\begin{remark}
Matrix-vector multiplication. A linear transformation is completely defined by its values on a generating set.
\end{remark}

\subsection{Linear transformation as a vector space}

\begin{remark}
The set of linear transformations from $V$ to $W$ (represented by $m\times n$ matrices) forms a vector space.
\end{remark}

\subsection{Composition of linear transformation and matrix multiplication}

\begin{remark}
Matrix multiplication.
\end{remark}

\begin{proposition} (properties of matrix mult.)

\end{proposition}

\begin{defin}
trace:
\end{defin}

\begin{proposition}~\\
a. \newline
b.
\end{proposition}

\subsection{Invertible transformations and matrices. Isomorphisms}

\begin{defin}
identity transformation \newline
left (right) invertible transformation and inverses \newline
invertible
\end{defin}

\begin{proposition} (alternative def. of invertible)

\end{proposition}

\begin{proposition}
(inverse of product and transposition)
a. \newline
b. 
\end{proposition}

\begin{proposition}
isomorphism preserves basis and vice versa
\end{proposition}

\begin{proposition}
invertible and equation
\end{proposition}

\subsection{Subspaces}
\begin{defin}~\\
subspace \newline
\indent null space/kernel of A $\ker A$ \newline
\indent range of a $\range A$ \newline
\indent linear span of ... $\mathscr{L}$
\end{defin}

\subsection{Applications to computer graphic}
wanna just skip...



\section{System of linear equations}






\end{document}